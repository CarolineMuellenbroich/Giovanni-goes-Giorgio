%  article.tex (Version 1.00, released 1 October 2013)
%  LaTeX source file to demonstrate formatting for manuscripts 
%  to be submitted to SPIE journals
%  based on the class file spieman.cls, which requires 
%  the following standard packages:  times.sty, float.sty, 
%  ifthen.sty, cite.sty, color.sty, setspace.sty
%
%  Special instructions are included in this file after the
%  symbol %>>>>
%  Numerous commands are commented out, but included to show how
%  to effect various options, for example, to print page numbers, etc.
%  This LaTeX source file is composed for LaTeX2e.

%  The following commands have been added in the SPIE class 
%  file (spieman.cls) and may not be understood in other classes:
%  \affiliations{}, \sup{},\supit{}, \authorinfo{}, \keywords{},
%  \linkable and \video
%  The bibliography style file is called spiejour.bst 
%  This article.tex file needs the following image files:
%  mcr3b.eps
%  fig2.eps
%  satellite.eps

\documentclass[12pt]{spieman}  %>>> 12pt font mandatory; use this for US letter paper 
%%\documentclass[a4paper,12pt]{spieman}  %>>> use this instead for A4 paper
%%\documentclass[nocompress,12pt]{spieman}  %>>> to avoid compression of citations
%% \addtolength{\voffset}{9mm}   %>>> moves text field down
%  The following command loads a graphics package to include images 
%  in the document. It may be necessary to specify a DVI driver option,
%  for example, [dvips], but that may be inappropriate for some LaTeX installations. 
%%\usepackage{amsmath}  %>>> for AMS math formatting, 
%  including bold Greek symbols
\usepackage[]{graphicx}
\usepackage{setspace}
\usepackage{tocloft}
\usepackage{units}
\usepackage{nicefrac}
\usepackage{color}
\usepackage{epstopdf}
\usepackage[globalcitecopy,labelstoglobalaux,sectionbib]{bibunits}
\usepackage[thinspace,thinqspace,squaren]{SIunits}
\usepackage{fancyhdr}
\usepackage{url}
\usepackage{wasysym}
%\usepackage{alltt}
%\renewcommand{\ttdefault}{txtt} 
\usepackage{array}
%\usepackage[version=3]{mhchem}
%\usepackage[T1]{fontenc}
%\usepackage{textcomp}
%\usepackage{verbatim} 
%\usepackage{mathpazo}
%\usepackage{amsmath}
%\usepackage{mathtools}
\usepackage{amssymb}
\usepackage{accents}
\usepackage{url}
\usepackage{amscd}
\usepackage{epsfig}
\usepackage{cite}
\usepackage{subfig}
\usepackage[globalcitecopy,labelstoglobalaux,sectionbib]{bibunits}
%\usepackage[sectionbib]{bibunits}
\usepackage{appendix}
\usepackage{slashbox}
%\usepackage{multirow}
%\usepackage{caption}

\title{Working title: the most amaz-zing SPIM and how we built it} 

%>>>> The author is responsible for formatting the 
%  author list and their affiliations.  Use \\ to force linebreaks.
%  The correspondence between each author and his/her address
%  can be indicated with a superscript, 
%  which is easily obtained with \supscr{}.
%>>>> After the abstract, the author must include
%  a list of keywords and contact information for the corresponding author.

\author{First Author,\supscr{a} Second Author,\supscr{a} Third Author,\supscr{b} Fourth Author\supscr{a,b}}

\affiliation{\supscrsm{a}European Laboratory for Non-linear Spectroscopy, University of Florence, Via Nello Carrara,1, Sesto Fiorentino (Firenze), Italy, 50019\\
\supscrsm{b}National Institute of Optics, National Research Council, Italy\\
\supscrsm{c}Department of Physics and Astronomy, University of Florence, Via Giovanni Sansone, 1, Sesto Fiorentino (Firenze), Italy, 50019}

%%%%%%%%%%%%%%%%%%%%%%%%%%%%%%%%%%%%%%%%%%%%%%%%%%%%%%%%%%%%% 
\renewcommand{\cftdotsep}{\cftnodots}
\cftpagenumbersoff{figure}
\cftpagenumbersoff{table} 
\begin{document} 
\maketitle 

%%%%%%%%%%%%%%%%%%%%%%%%%%%%%%%%%%%%%%%%%%%%%%%%%%%%%%%%%%%%% 
\begin{abstract}
200 words limit. no numerical references presenting concisely the objectives, methodology used, results obtained, and their significance.
\end{abstract}

%>>>> Include a list of up to six keywords after the abstract
\keywords{Light sheet microscopy, Big data, Optical clearing, Data management, whole brain imaging, rolling shutter, 7,8.}

%>>>> Include contact information for corresponding author
{\noindent \footnotesize{\bf Address all correspondence to}: First author, University Name, Faculty Group, Department, Street Address, City, Country, Postal Code; Tel: +1 555-555-5555; Fax: +1 555-555-5556; E-mail:  \linkable{myemail@university.edu} }
%%%%%%%%%%%%%%%%%%%%%%%%%%%%%%%%%%%%%%%%%%%%%%%%%%%%%%%%%%%%%

\begin{spacing}{2}   % use double spacing for rest of manuscript

%%%%%%%%%%%%%%%%%%%%%%%%%%%%%%%%%%%%%%%%%%%%%%%%%%%%%%%%%%%%%
\section{Introduction}
\label{sect:intro}  % \label{} allows reference to this section
Here will be an introduction to whole brain imaging the challenges it is facing and how light sheet microscopy is addressing those challenges. Points to raise: briefly optical clearing, big data generation and management, image quality degradation necessitating rolling shutter, refocusing?, double sided illumination, maybe Bessel beam illumination, etc. 

\section{Optical Setup}
\subsection{Optical path}
Here we describe our optical setup, the components we use etc

\subsection{Alignment}
and maybe also some details on who we aligned everything. we have our sample mirror with the hole in it, the shear plate, periscopes etc...

\subsection{Sample chamber}
the sample chamber is probably of particular interest, how was it designed and made, how do we protect the objectives, stop it from leaking, soft connections and hard connections. 


\section{Software Management}
Here we describe all things murmex, components and their communication/ interaction. how is the rolling shutter timed with galvos, triggers and clocks etc. 

\section{Data Management}
Here we describe our data pipeline: data production rate as function of frame rate and image size, stack size, tomo size, SSD space, transfer to NAS digital downsampling, and tiff compression, NAS to CINECA, 10 gigabit connection etc.

\section{Data}
some pretty images to prove that it all worked. 

\section{Conclusion}
summary

\section{Outlook}
where are we going next with this? 

%%-------------
   %\begin{figure}
   %\begin{center}
   %\begin{tabular}{c}
   %\includegraphics[height=5.5cm]{mcr3b.eps}
   %\end{tabular}
   %\end{center}
   %\caption 
   %{ \label{fig:example} %>> use \label inside caption to get Fig. number with \ref{}
%Example of a figure caption. } 
   %\end{figure} 
%
%%------------- 
   %\begin{figure}
   %\begin{center}
   %\begin{tabular}{c}
   %\includegraphics[height=5.5cm]{fig2.eps}  % fig2 includes two images 
     %\\
     %(a) \hspace{5.1cm} (b)
   %\end{tabular}
   %\end{center}
   %\caption 
   %{ \label{fig:example2} %>> use \label inside caption to get Fig. number with \ref{}
%Example of a figure containing multiple images: (a) sun and (b) blob. Figures containing multiple images must be submitted to SPIE as a single image file.} 
   %\end{figure} 

%%%%%%%%%%%%%%%%%%%%%%%%%%%%%%%%%%%%%%%%%%%%%%%%%%%%%%%%%%%%%
\acknowledgments 
Human Brain Project tutta la vita. 

%%%%%%%%%%%%%%%%%%%%%%%%%%%%%%%%%%%%%%%%%%%%%%%%%%%%%%%%%%%%%
%%%%% References %%%%%

%\bibliography{report2}   %>>>> bibliography data in report.bib
\bibliographystyle{spiejour}   %>>>> makes bibtex use spiejour.bst
\begin{thebibliography}{0}

\bibitem{Silvestri2012} Silvestri, L., Bria, A., Sacconi, L., Iannello, G. \& Pavone, F. S., XY. 	\emph{Opt Express} \textbf{20}, 20582-20598 (2012).

\end{thebibliography}


\listoffigures
\listoftables

\end{spacing}
\end{document} 
